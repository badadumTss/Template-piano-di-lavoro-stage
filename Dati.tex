%----------------------------------------------------------------------------------------
%   USEFUL COMMANDS
%----------------------------------------------------------------------------------------

\newcommand{\dipartimento}{Dipartimento di Matematica ``Tullio Levi-Civita''}

%----------------------------------------------------------------------------------------
% 	USER DATA
%----------------------------------------------------------------------------------------

% Data di approvazione del piano da parte del tutor interno; nel formato GG Mese AAAA
% compilare inserendo al posto di GG 2 cifre per il giorno, e al posto di 
% AAAA 4 cifre per l'anno
\newcommand{\dataApprovazione}{Data}

% Dati dello Studente
\newcommand{\nomeStudente}{Luca}
\newcommand{\cognomeStudente}{Zaninotto}
\newcommand{\matricolaStudente}{1187304}
\newcommand{\emailStudente}{luca.zaninotto.1@studenti.unipd.it}
\newcommand{\telStudente}{Numero}

% Dati del Tutor Aziendale
\newcommand{\nomeTutorAziendale}{Fabio}
\newcommand{\cognomeTutorAziendale}{Pallaro}
\newcommand{\emailTutorAziendale}{f.pallaro@synclab.it}
\newcommand{\telTutorAziendale}{Numero}
\newcommand{\ruoloTutorAziendale}{Manager Responsabile Sede}

% Dati dell'Azienda
\newcommand{\ragioneSocAzienda}{SyncLab s.r.l}
\newcommand{\indirizzoAzienda}{Galleria Spagna 28, Padova (PD)}
\newcommand{\sitoAzienda}{https://www.synclab.it}
\newcommand{\emailAzienda}{info@synclab.it}
\newcommand{\partitaIVAAzienda}{Manager Responsabile Sede}

% Dati del Tutor Interno (Docente)
\newcommand{\titoloTutorInterno}{Prof.}
\newcommand{\nomeTutorInterno}{Nome}
\newcommand{\cognomeTutorInterno}{Cognome}

\newcommand{\prospettoSettimanale}{
  \begin{itemize}
  \item \textbf{Prima Settimana (40 ore)}
    \begin{itemize}
    \item Incontro con persone coinvolte nel progetto per
      discutere i requisiti e le richieste relativamente al
      sistema da sviluppare;
    \item Verifica credenziali e strumenti di lavoro assegnati;
    \item Studio dell'implementazione classica di blockchain per la
      gestione di criptovalute;
    \end{itemize}
  \item \textbf{Seconda Settimana (40 ore)}
    \begin{itemize}
    \item Studio dell'estensione ethereum;
    \end{itemize}
  \item \textbf{Terza Settimana (40 ore)}
    \begin{itemize}
    \item Studio del linguaggio \textit{Solidity} per la definizione
      di \textit{smart contract};
    \end{itemize}
  \item \textbf{Quarta Settimana (40 ore)}
    \begin{itemize}
    \item Studio degli standard per la gestione di NFT;
    \end{itemize}
  \item \textbf{Quinta Settimana (40 ore)}
    \begin{itemize}
    \item Analisi del problema e del dominio applicativo;
    \item Progettazione di una soluzione al problema e metodologia di
      testing;
    \end{itemize}
  \item \textbf{Sesta Settimana (40 ore)}
    \begin{itemize}
    \item Conclusione progettazione e stesura della documentazione
      relativa;
    \item Codifica della soluzione;
    \end{itemize}
  \item \textbf{Settima Settimana (40 ore)}
    \begin{itemize}
    \item Conclusione codifica e stesura dei test;
    \item Stesura della documentazione relativa al periodo di
      codifica;
    \end{itemize}
  \item \textbf{Ottava Settimana (40 ore)}
    \begin{itemize}
    \item Collaudo della soluzione, stesura documentazione finale;
    \item Incontro di presentazione della soluzione con gli
      stakeholders
    \item Live demo del lavoro di stage
    \end{itemize}
  \end{itemize}
}

% Indicare il totale complessivo (deve essere compreso tra le 300 e le 320 ore)
\newcommand{\totaleOre}{300}

\newcommand{\obiettiviObbligatori}{
\item \underline{\textit{O01}}: Studio della tecnologia blockchain;
\item \underline{\textit{O02}}: Studio dell'estensione ethereum;
\item \underline{\textit{O03}}: Studio del linguaggio solidity;
\item \underline{\textit{O04}}: Studio degli standard per la gestione di non fungible token (NFT);
\item \underline{\textit{O05}}: implementazione di contratti in catena per la gestione di NFT;
}

\newcommand{\obiettiviDesiderabili}{
\item \underline{\textit{D01}}: primo obiettivo;
}

\newcommand{\obiettiviFacoltativi}{
\item \underline{\textit{F01}}: primo obiettivo;
}
